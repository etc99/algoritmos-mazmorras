\apendice{Sostenibilización curricular}

\section{Introducción}
El documento ``Directrices para la introducción de la Sostenibilidad en el Curriculum'' aprobado por la CRUE establece una serie de objetivos clave para la integración de la sostenibilidad en el currículo de las universidades españolas. Estos objetivos tienen como propósito fundamental fomentar un desarrollo humano sostenible a través de la educación superior. A continuación, se detallan los principales objetivos planteados:

\section{Integración Transversal de la Sostenibilidad}
\textbf{Objetivo:} Incluir la sostenibilidad como un eje transversal en todas las titulaciones universitarias, asegurando que todos los estudiantes adquieran competencias en este ámbito, independientemente de su área de estudio.\\
\textbf{Acciones:} Se promoverá la revisión y actualización de los planes de estudio para incluir contenidos específicos y transversales sobre sostenibilidad, adaptados a cada disciplina.

\section{Desarrollo de Competencias para la Sostenibilidad}
\textbf{Objetivo:} Desarrollar en los estudiantes las competencias necesarias para identificar y resolver problemas socioambientales desde una perspectiva sistémica y ética.\\
\textbf{Acciones:} Los planes de estudio deberán contemplar la formación en competencias como el pensamiento crítico, la gestión sostenible de recursos, la participación en procesos comunitarios y la aplicación de principios éticos en la vida personal y profesional.

\section{Fomento de la Investigación y Docencia Sostenible}
\textbf{Objetivo:} Promover la investigación y la docencia que integren los principios del desarrollo sostenible, asegurando que la actividad académica contribuya a la solución de problemas ambientales y sociales.\\
\textbf{Acciones:} Se incentivará la realización de proyectos de investigación y la impartición de cursos que aborden la sostenibilidad desde diversas perspectivas, facilitando la formación continua y el aprendizaje a lo largo de la vida.

\section{Sensibilización y Participación de la Comunidad Universitaria}
\textbf{Objetivo:} Involucrar a toda la comunidad universitaria en el compromiso con la sostenibilidad, fomentando la participación activa en la implementación de políticas y acciones sostenibles dentro y fuera del campus.\\
\textbf{Acciones:} Se promoverán actividades extracurriculares, seminarios, talleres y jornadas que sensibilicen a estudiantes, profesores y personal administrativo sobre la importancia de la sostenibilidad y su aplicación en la vida diaria.

\section{Establecimiento de Mecanismos de Evaluación y Mejora Continua}
\textbf{Objetivo:} Implementar sistemas de evaluación que aseguren la calidad y efectividad de las acciones educativas en sostenibilidad, permitiendo una mejora continua de los programas académicos y las prácticas institucionales.\\
\textbf{Acciones:} Se incluirán criterios de sostenibilidad en los sistemas de evaluación de la calidad universitaria y en la evaluación del profesorado, garantizando que tanto la docencia como la investigación se alineen con los principios del desarrollo sostenible.

\section{Creación de Redes y Plataformas de Colaboración}
\textbf{Objetivo:} Establecer redes de colaboración entre universidades y otras entidades para intercambiar experiencias y buenas prácticas en la integración de la sostenibilidad en el currículo académico.\\
\textbf{Acciones:} Se fomentará la creación de grupos de trabajo interuniversitarios y la participación en plataformas estatales e internacionales que faciliten el intercambio de conocimientos y la cooperación en proyectos de sostenibilidad.

Estos objetivos buscan no solo la formación de profesionales capacitados para enfrentar los desafíos actuales y futuros, sino también la creación de una cultura universitaria comprometida con la sostenibilidad y el desarrollo humano integral. La implementación efectiva de estas directrices requiere del compromiso institucional y la participación activa de toda la comunidad universitaria.\cite{crue2020}

