\capitulo{1}{Introducción}

La generación procedimental, es la técnica que a través de algoritmos, permite crear contenido para videojuegos de forma autónoma o de la mano de un diseñador. 
Usar esta técnica permite optimizar o crear videojuegos que tengan la posibilidad de crear nuevos escenarios completamente nuevos para cada jugador.

Esta técnica forma parte de algunos videojuegos que llevan publicados más de tres décadas, y a día de hoy es una técnica muy popular que es utilizada en juegos muy reconocidos. Uno de sus usos más populares y reconocidos es para la creación de mapas para videojuegos o incluso juegos de mesa. 

Es por ello que para este trabajo de fin de grado, se van a explorar distintos algoritmos para poder probar esta técnica, de forma que se obtengan mapas navegables como lo serían en un videojuego. Estos mapas van a tener la forma de un laberinto y van a ser generados en un servidor, replicando la arquitectura que se tendría en un estudio de videojuegos real.


